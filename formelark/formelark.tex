\documentclass[12pt]{article}
\usepackage[utf8]{inputenc}
\usepackage[T1]{fontenc}
\usepackage{amsmath}
\usepackage{amsthm}
\usepackage{amsfonts}
\usepackage{amssymb}
\usepackage{mathtools}
\usepackage{enumerate}
\usepackage{physics}
\usepackage[top=0.4in, bottom=0.4in, left=0.4in, right=0.4in]{geometry}

\thispagestyle{empty}

\begin{document}
\section{Energi i termofysikk}
\subsection{Ideell gass}
For lav-tetthets gasser gjelder
\begin{align*}
  PV = NkT = nRT
\end{align*}
\subsection{Ekvipartisjon av energi}
Gjelder for alle energiformer der formelen er en kvadratisk funksjon av koordinat-
eller hastighetskomponent. Dersom det i et system er $N$ molekyler med $f$
frihetsgrader hver, ingen ikke-kvadratiske temperaturavhengige former for energi
er den midlere (totale for store $N$) termiske energien gitt ved
\begin{align*}
  U_\text{termisk} = N f \frac{1}{2}kT
\end{align*}
Tar ikke hensyn til for eksempel hvileenergi, så tryggest å bruke for endringer
i energi med endringer i temperatur, og unngå tilfeller med faseoverganger
eller andre reaksjoner der kjemiske bindinger frigjør/krever energi.

\subsubsection{Telling av frihetsgrader}
Telling av frihetsgrader. Monoatomisk gass: Kun translasjonell bevegelse i tre
dimensjoner gir frihetsgrader, så $f = 3$. Diatomisk gass: Translasjonell
bevegelse i 3D + rotasjon om to akser (aksen gjennom lengden av molekylet
teller ikke pga. kvantemekanikk), så $f = 5$. Polyatomiske molekyler kan som
regel roteres om tre akser, som gir det siste rotasjonsbidraget. Vibrasjon
gir både kinetisk og potensiell energi bidrag, altså øker frihetsgraden med to.
Ved romtemperatur bidrar ikke vibrasjon til termisk energi, ved høyere temperaturer
kommer dette bidraget inn.

I et fast stoff kan atomer vibrere i tre ortogonale retninger, som gir 6 frihetsgrader.
Noen av disse kan "fryses ut" ved romtemperatur. Væsker er som regel mer kompliserte,
ekvipartisjonsteoremet gjelder for translasjonell bevegelse, men ikke for andre
bidrag siden de er ikke-kvadratiske.

\subsection{Varme og arbeid}
Overføringer av energi klassifiseres på to måter: \newline \noindent
\textbf{Varme}: overføring av energi fra et objekt til et annet på grunn av innbyrdes forskjell i temperatur. Ledning (molekylær
kontakt), konveksjon (bevegelse av områder av gass/væske) og elektromagnetisk stråling.\newline \noindent
\textbf{Arbeid} er enhver annen overføring av energi inn eller ut av et system.
\newline \noindent
Endringen i energi for et system er summen av varme og arbeid (termodynamikkens første lov)
\begin{align*}
  \Delta{U} = Q + W
\end{align*}
\subsection{Kompresjonsarbeid}
Arbeid som behøves for å redusere et volum av gass (trykk $P$) $\Delta V$ ved
\textbf{kvasistatisk} kompresjon (gassen rekker hele tiden å justere til likevekt).
\begin{align*}
  W = -P \Delta{V} \quad (\text{ved } P \neq P(V))\qquad W = - \int_{V_i}^{V_f} P(V) \dd V \quad (\text{ved } P = P(V))
\end{align*}
To idealiserte kompresjonstyper for ideell gass: \textbf{isotermal kompresjon}
(så sakte at temperatur i gassen er uendret) og \textbf{adiabatisk kompresjon}
(så kjapp at ingen varme slipper ut av systemet i prosessen). \newline \noindent
\textbf{Isotermal kompresjon}: Kvasistatisk, så
\begin{align*}
  W = -\int_{V_i}^{V_f} P \dd V = -NkT \int_{V_i}^{V_f} \frac{1}{V} \dd V = NkT \ln{\frac{V_i}{V_f}}
\end{align*}
Samme energi slipper ut ved varme (vises med første lov).\newline \noindent
\textbf{Adiabatisk kompresjon}: Ingen varme unnslipper, $\Delta U = Q + W = W$.
Bruk ekvipartisjon $\dd U = \frac{f}{2} N k \dd T$ og kvasistatisk kompresjon,
bruk ideell gass lov for trykket og løs separabel difflign. Får (med $\gamma = f(f+2)$)
\begin{align*}
  V T^{f/2} = \text{constant} \overset{\text{Ideell gass lov}}{\implies} V^\gamma P = \text{constant}
\end{align*}
\subsection{Varmekapasiteter}
Varmekapasitet defineres som $C = Q/\Delta{T}$ (varme per grad temperaturøkning). Tvetydig,
avhenger av omstendigheter (mengde stoff/gass, arbeid osv.). Derfor har man
varmekapasitet med $W = 0$ (som regel er $V$ da konstant, ingen kompressjon):
\begin{align*}
  C_V = \left( \frac{\Delta U}{\Delta T} \right)_V = \left( \pdv{U}{T}\right)_V
\end{align*}
Ofte ekspanderer gasser når de varmes opp, da tapes energi ved arbeid på omgivelsene.
Dersom det er konstant trykk i omgivelsene er
\begin{align*}
  C_P = \left( \frac{\Delta U - (-P\Delta V)}{\Delta T} \right)_P =  \left( \pdv{U}{T}\right)_P + P\left( \pdv{V}{T}\right)_P
\end{align*} \newline \noindent
\textbf{Latent varme:} Ved faseoverganger øker ikke temperaturen ved tilførsel av energi. Da
er varmekapasiteten per def uendelig: $C = Q/\Delta{T} = Q/0 = \infty$. Energien som kreves
for å fullføre faseoverangen kalles den latente varmen $L = Q/m$, og er varmen per
masse som kreves.\newline \noindent
\textbf{Entalpi:} Entalpi er energien til et system + ekspansjonsarbeidet som
krevdes for å putte det i det systemet det er i: $H = U + PV$. Observer at
$C_P = \left( \pdv{H}{T} \right)_P$.
\subsection{Prosesshastigheter (rates of processes)}
Orker ikke dette nå.
\section{Termodynamikkens andre lov}

\section{Interaksjoner og implikasjoner}

\section{Motorer og kjølemaskiner}

\section{Fri energi og kjemisk termodynamikk}

\section{Boltzmann statistikk}

\section{Kvantestatistikk}

\section{Konstanter og annet nyttig}
\end{document}
